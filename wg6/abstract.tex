\documentclass{easychair}
%% ODER: format ==         = "\mathrel{==}"
%% ODER: format /=         = "\neq "
%
%
\makeatletter
\@ifundefined{lhs2tex.lhs2tex.sty.read}%
  {\@namedef{lhs2tex.lhs2tex.sty.read}{}%
   \newcommand\SkipToFmtEnd{}%
   \newcommand\EndFmtInput{}%
   \long\def\SkipToFmtEnd#1\EndFmtInput{}%
  }\SkipToFmtEnd

\newcommand\ReadOnlyOnce[1]{\@ifundefined{#1}{\@namedef{#1}{}}\SkipToFmtEnd}
\usepackage{amstext}
\usepackage{amssymb}
\usepackage{stmaryrd}
\DeclareFontFamily{OT1}{cmtex}{}
\DeclareFontShape{OT1}{cmtex}{m}{n}
  {<5><6><7><8>cmtex8
   <9>cmtex9
   <10><10.95><12><14.4><17.28><20.74><24.88>cmtex10}{}
\DeclareFontShape{OT1}{cmtex}{m}{it}
  {<-> ssub * cmtt/m/it}{}
\newcommand{\texfamily}{\fontfamily{cmtex}\selectfont}
\DeclareFontShape{OT1}{cmtt}{bx}{n}
  {<5><6><7><8>cmtt8
   <9>cmbtt9
   <10><10.95><12><14.4><17.28><20.74><24.88>cmbtt10}{}
\DeclareFontShape{OT1}{cmtex}{bx}{n}
  {<-> ssub * cmtt/bx/n}{}
\newcommand{\tex}[1]{\text{\texfamily#1}}	% NEU

\newcommand{\Sp}{\hskip.33334em\relax}


\newcommand{\Conid}[1]{\mathit{#1}}
\newcommand{\Varid}[1]{\mathit{#1}}
\newcommand{\anonymous}{\kern0.06em \vbox{\hrule\@width.5em}}
\newcommand{\plus}{\mathbin{+\!\!\!+}}
\newcommand{\bind}{\mathbin{>\!\!\!>\mkern-6.7mu=}}
\newcommand{\rbind}{\mathbin{=\mkern-6.7mu<\!\!\!<}}% suggested by Neil Mitchell
\newcommand{\sequ}{\mathbin{>\!\!\!>}}
\renewcommand{\leq}{\leqslant}
\renewcommand{\geq}{\geqslant}
\usepackage{polytable}

%mathindent has to be defined
\@ifundefined{mathindent}%
  {\newdimen\mathindent\mathindent\leftmargini}%
  {}%

\def\resethooks{%
  \global\let\SaveRestoreHook\empty
  \global\let\ColumnHook\empty}
\newcommand*{\savecolumns}[1][default]%
  {\g@addto@macro\SaveRestoreHook{\savecolumns[#1]}}
\newcommand*{\restorecolumns}[1][default]%
  {\g@addto@macro\SaveRestoreHook{\restorecolumns[#1]}}
\newcommand*{\aligncolumn}[2]%
  {\g@addto@macro\ColumnHook{\column{#1}{#2}}}

\resethooks

\newcommand{\onelinecommentchars}{\quad-{}- }
\newcommand{\commentbeginchars}{\enskip\{-}
\newcommand{\commentendchars}{-\}\enskip}

\newcommand{\visiblecomments}{%
  \let\onelinecomment=\onelinecommentchars
  \let\commentbegin=\commentbeginchars
  \let\commentend=\commentendchars}

\newcommand{\invisiblecomments}{%
  \let\onelinecomment=\empty
  \let\commentbegin=\empty
  \let\commentend=\empty}

\visiblecomments

\newlength{\blanklineskip}
\setlength{\blanklineskip}{0.66084ex}

\newcommand{\hsindent}[1]{\quad}% default is fixed indentation
\let\hspre\empty
\let\hspost\empty
\newcommand{\NB}{\textbf{NB}}
\newcommand{\Todo}[1]{$\langle$\textbf{To do:}~#1$\rangle$}

\EndFmtInput
\makeatother
%
%
%
%
%
%
% This package provides two environments suitable to take the place
% of hscode, called "plainhscode" and "arrayhscode". 
%
% The plain environment surrounds each code block by vertical space,
% and it uses \abovedisplayskip and \belowdisplayskip to get spacing
% similar to formulas. Note that if these dimensions are changed,
% the spacing around displayed math formulas changes as well.
% All code is indented using \leftskip.
%
% Changed 19.08.2004 to reflect changes in colorcode. Should work with
% CodeGroup.sty.
%
\ReadOnlyOnce{polycode.fmt}%
\makeatletter

\newcommand{\hsnewpar}[1]%
  {{\parskip=0pt\parindent=0pt\par\vskip #1\noindent}}

% can be used, for instance, to redefine the code size, by setting the
% command to \small or something alike
\newcommand{\hscodestyle}{}

% The command \sethscode can be used to switch the code formatting
% behaviour by mapping the hscode environment in the subst directive
% to a new LaTeX environment.

\newcommand{\sethscode}[1]%
  {\expandafter\let\expandafter\hscode\csname #1\endcsname
   \expandafter\let\expandafter\endhscode\csname end#1\endcsname}

% "compatibility" mode restores the non-polycode.fmt layout.

\newenvironment{compathscode}%
  {\par\noindent
   \advance\leftskip\mathindent
   \hscodestyle
   \let\\=\@normalcr
   \let\hspre\(\let\hspost\)%
   \pboxed}%
  {\endpboxed\)%
   \par\noindent
   \ignorespacesafterend}

\newcommand{\compaths}{\sethscode{compathscode}}

% "plain" mode is the proposed default.
% It should now work with \centering.
% This required some changes. The old version
% is still available for reference as oldplainhscode.

\newenvironment{plainhscode}%
  {\hsnewpar\abovedisplayskip
   \advance\leftskip\mathindent
   \hscodestyle
   \let\hspre\(\let\hspost\)%
   \pboxed}%
  {\endpboxed%
   \hsnewpar\belowdisplayskip
   \ignorespacesafterend}

\newenvironment{oldplainhscode}%
  {\hsnewpar\abovedisplayskip
   \advance\leftskip\mathindent
   \hscodestyle
   \let\\=\@normalcr
   \(\pboxed}%
  {\endpboxed\)%
   \hsnewpar\belowdisplayskip
   \ignorespacesafterend}

% Here, we make plainhscode the default environment.

\newcommand{\plainhs}{\sethscode{plainhscode}}
\newcommand{\oldplainhs}{\sethscode{oldplainhscode}}
\plainhs

% The arrayhscode is like plain, but makes use of polytable's
% parray environment which disallows page breaks in code blocks.

\newenvironment{arrayhscode}%
  {\hsnewpar\abovedisplayskip
   \advance\leftskip\mathindent
   \hscodestyle
   \let\\=\@normalcr
   \(\parray}%
  {\endparray\)%
   \hsnewpar\belowdisplayskip
   \ignorespacesafterend}

\newcommand{\arrayhs}{\sethscode{arrayhscode}}

% The mathhscode environment also makes use of polytable's parray 
% environment. It is supposed to be used only inside math mode 
% (I used it to typeset the type rules in my thesis).

\newenvironment{mathhscode}%
  {\parray}{\endparray}

\newcommand{\mathhs}{\sethscode{mathhscode}}

% texths is similar to mathhs, but works in text mode.

\newenvironment{texthscode}%
  {\(\parray}{\endparray\)}

\newcommand{\texths}{\sethscode{texthscode}}

% The framed environment places code in a framed box.

\def\codeframewidth{\arrayrulewidth}
\RequirePackage{calc}

\newenvironment{framedhscode}%
  {\parskip=\abovedisplayskip\par\noindent
   \hscodestyle
   \arrayrulewidth=\codeframewidth
   \tabular{@{}|p{\linewidth-2\arraycolsep-2\arrayrulewidth-2pt}|@{}}%
   \hline\framedhslinecorrect\\{-1.5ex}%
   \let\endoflinesave=\\
   \let\\=\@normalcr
   \(\pboxed}%
  {\endpboxed\)%
   \framedhslinecorrect\endoflinesave{.5ex}\hline
   \endtabular
   \parskip=\belowdisplayskip\par\noindent
   \ignorespacesafterend}

\newcommand{\framedhslinecorrect}[2]%
  {#1[#2]}

\newcommand{\framedhs}{\sethscode{framedhscode}}

% The inlinehscode environment is an experimental environment
% that can be used to typeset displayed code inline.

\newenvironment{inlinehscode}%
  {\(\def\column##1##2{}%
   \let\>\undefined\let\<\undefined\let\\\undefined
   \newcommand\>[1][]{}\newcommand\<[1][]{}\newcommand\\[1][]{}%
   \def\fromto##1##2##3{##3}%
   \def\nextline{}}{\) }%

\newcommand{\inlinehs}{\sethscode{inlinehscode}}

% The joincode environment is a separate environment that
% can be used to surround and thereby connect multiple code
% blocks.

\newenvironment{joincode}%
  {\let\orighscode=\hscode
   \let\origendhscode=\endhscode
   \def\endhscode{\def\hscode{\endgroup\def\@currenvir{hscode}\\}\begingroup}
   %\let\SaveRestoreHook=\empty
   %\let\ColumnHook=\empty
   %\let\resethooks=\empty
   \orighscode\def\hscode{\endgroup\def\@currenvir{hscode}}}%
  {\origendhscode
   \global\let\hscode=\orighscode
   \global\let\endhscode=\origendhscode}%

\makeatother
\EndFmtInput
%
%
\ReadOnlyOnce{agda.fmt}%


\RequirePackage[T1]{fontenc}
\RequirePackage[utf8x]{inputenc}
\RequirePackage{ucs}
\RequirePackage{amsfonts}

\providecommand\mathbbm{\mathbb}

% TODO: Define more of these ...
\DeclareUnicodeCharacter{737}{\textsuperscript{l}}
\DeclareUnicodeCharacter{8718}{\ensuremath{\blacksquare}}
\DeclareUnicodeCharacter{8759}{::}
\DeclareUnicodeCharacter{9669}{\ensuremath{\triangleleft}}
\DeclareUnicodeCharacter{8799}{\ensuremath{\stackrel{\scriptscriptstyle ?}{=}}}
\DeclareUnicodeCharacter{10214}{\ensuremath{\llbracket}}
\DeclareUnicodeCharacter{10215}{\ensuremath{\rrbracket}}

% TODO: This is in general not a good idea.
\providecommand\textepsilon{$\epsilon$}
\providecommand\textmu{$\mu$}


%Actually, varsyms should not occur in Agda output.

% TODO: Make this configurable. IMHO, italics doesn't work well
% for Agda code.

\renewcommand\Varid[1]{\mathord{\textsf{#1}}}
\let\Conid\Varid
\newcommand\Keyword[1]{\textsf{\textbf{#1}}}
\EndFmtInput






 







%%format [⇒] = "[\rightarrow]"
%%format ⊢ = "\vdash"
%%format _⊢_ = "\_\vdash\_"























\usepackage{amssymb}

% use this if you have a long article and want to create an index
% \usepackage{makeidx}

% In order to save space or manage large tables or figures in a
% landcape-like text, you can use the rotating and pdflscape
% packages. Uncomment the desired from the below.
%
% \usepackage{rotating}
% \usepackage{pdflscape}

% Some of our commands for this guide.
%

%\makeindex

%% Front Matter
%%
% Regular title as in the article class.
%
\title{Containers in Higher Kinds}

% Authors are joined by \and. Their affiliations are given by \inst, which indexes
% into the list defined using \institute
%
\author{
  Thorsten Altenkirch \and
  Zhili Tian
}

\institute{
  School of Computer Science, University of Nottingham, UK\\
  \email{\{psztxa,psxzt8\}@nottingham.ac.uk}
}

%  \authorrunning{} has to be set for the shorter version of the authors' names;
% otherwise a warning will be rendered in the running heads. When processed by
% EasyChair, this command is mandatory: a document without \authorrunning
% will be rejected by EasyChair

\authorrunning{Altenkirch, Tian}

% \titlerunning{} has to be set to either the main title or its shorter
% version for the running heads. When processed by
% EasyChair, this command is mandatory: a document without \titlerunning
% will be rejected by EasyChair
\titlerunning{Containers in Higher Kinds}

\begin{document}

\maketitle
Strictly positive types can be represented as containers \cite{containers}, that is 
\ensuremath{\Conid{S}\;\mathbin{:}\;\Conid{Set}} and a family of positions \ensuremath{\Conid{P}\;\mathbin{:}\;\Conid{S}\;\rightarrow\;\Conid{Set}} giving rise to a
functor \ensuremath{\Conid{S}\;\lhd\;\Conid{P}\;\mathbin{:}\;\Conid{Set}\;\Rightarrow\;\Conid{Set}} given by \ensuremath{(\Conid{S}\;\lhd\;\Conid{P})\;\Conid{X}\;\mathrel{=}\;\Sigma\;\Varid{s}\;\mathbin{:}\;\Conid{S}\;.\;\Conid{P}\;\Varid{s}\;\rightarrow\;\Conid{X}}.
Every container has an inital algebra (The W-type) and a terminal
coalgebra (The M-type). However, there are types which don't fit into
this scheme, an example is the type \ensuremath{\Conid{Bush}} which can be defined
coinductively:
\begin{hscode}\SaveRestoreHook
\column{B}{@{}>{\hspre}l<{\hspost}@{}}%
\column{3}{@{}>{\hspre}l<{\hspost}@{}}%
\column{5}{@{}>{\hspre}l<{\hspost}@{}}%
\column{E}{@{}>{\hspre}l<{\hspost}@{}}%
\>[B]{}\Keyword{record}\;\Conid{Bush}\;(\Conid{A}\;\mathbin{:}\;\Conid{Set})\;\mathbin{:}\;\Conid{Set}\;\Keyword{where}{}\<[E]%
\\
\>[B]{}\hsindent{3}{}\<[3]%
\>[3]{}\Keyword{coinductive}{}\<[E]%
\\
\>[B]{}\hsindent{3}{}\<[3]%
\>[3]{}\Keyword{field}{}\<[E]%
\\
\>[3]{}\hsindent{2}{}\<[5]%
\>[5]{}\Varid{head}\;\mathbin{:}\;\Conid{A}{}\<[E]%
\\
\>[3]{}\hsindent{2}{}\<[5]%
\>[5]{}\Varid{tail}\;\mathbin{:}\;\Conid{Bush}\;(\Conid{Bush}\;\Conid{A}){}\<[E]%
\ColumnHook
\end{hscode}\resethooks
This type is isomorphic to the type of functions of binary trees, that
is \ensuremath{(\Conid{BT}\;\rightarrow\;\Conid{A})\;\cong\;\Conid{Bush}\;\Conid{A}} where \ensuremath{\Conid{BT}} is defined inductively as the
initial algebra of \ensuremath{\Conid{F}\;\Conid{X}\;\mathrel{=}\;\Varid{1}\;\uplus\;\Conid{X}\;\times\;\Conid{X}}, see \cite{altenkirch2001representations}.

Hence we want to define a notion of containers in higher kinds which
model strictly positive functors like the functor giving rise to
\ensuremath{\Conid{Bush}} which is:
\begin{hscode}\SaveRestoreHook
\column{B}{@{}>{\hspre}l<{\hspost}@{}}%
\column{E}{@{}>{\hspre}l<{\hspost}@{}}%
\>[B]{}\Conid{B}\;\mathbin{:}\;(\Conid{Set}\;\rightarrow\;\Conid{Set})\;\rightarrow\;\Conid{Set}\;\rightarrow\;\Conid{Set}{}\<[E]%
\\
\>[B]{}\Conid{B}\;\Conid{F}\;\Conid{X}\;\mathrel{=}\;\Conid{X}\;\times\;\Conid{F}\;(\Conid{F}\;\Conid{X}){}\<[E]%
\ColumnHook
\end{hscode}\resethooks

We define a notion of higher container \ensuremath{\Conid{HCont}\;\mathbin{:}\;\Conid{Ty}\;\rightarrow\;\Conid{Set}} where \ensuremath{\Conid{Ty}}
are just the types of simply typed $\lambda$-calculus with one base
type \ensuremath{\Varid{set}}. \ensuremath{\Conid{HCont}} is just a special case of \ensuremath{\Conid{HCont-NF}\;\mathbin{:}\;\Conid{Con}\;\rightarrow\;\Conid{Ty}\;\rightarrow\;\Conid{Set}} where \ensuremath{\Conid{Con}} are the contexts of simply typed $\lambda$-calculus
\ensuremath{\Conid{HCont}\;\Conid{A}\;\mathrel{=}\;\Conid{HCont-NF}\;\bullet\;\Conid{A}}. We also use \ensuremath{\Conid{Var}\;\mathbin{:}\;\Conid{Con}\;\rightarrow\;\Conid{Ty}\;\rightarrow\;\Conid{Set}} for the
typed de Bruijn variables.

The definition of \ensuremath{\Conid{HCont-NF}} is straightforward:
\begin{hscode}\SaveRestoreHook
\column{B}{@{}>{\hspre}l<{\hspost}@{}}%
\column{3}{@{}>{\hspre}l<{\hspost}@{}}%
\column{7}{@{}>{\hspre}l<{\hspost}@{}}%
\column{E}{@{}>{\hspre}l<{\hspost}@{}}%
\>[B]{}\Keyword{data}\;\Conid{HCont-NF}\;\Keyword{where}{}\<[E]%
\\
\>[B]{}\hsindent{3}{}\<[3]%
\>[3]{}\Varid{lam}\;\mathbin{:}\;\Conid{HCont-NF}\;(\Gamma\;\rhd\;\Conid{A})\;\Conid{B}\;\rightarrow\;\Conid{HCont-NF}\;\Gamma\;(\Conid{A}\;\Rightarrow\;\Conid{B}){}\<[E]%
\\
\>[B]{}\hsindent{3}{}\<[3]%
\>[3]{}\Varid{ne}\;{}\<[7]%
\>[7]{}\mathbin{:}\;\Conid{HCont-NE}\;\Gamma\;\rightarrow\;\Conid{HCont-NF}\;\Gamma\;\Varid{set}{}\<[E]%
\ColumnHook
\end{hscode}\resethooks
This is defined mutually with \ensuremath{\Conid{HCont-NE}\;\mathbin{:}\;\Conid{Con}\;\rightarrow\;\Conid{Set}} which represent
the type expressions in a given context, and also \ensuremath{\Conid{HCont-SP}\;\mathbin{:}\;\Conid{Con}\;\rightarrow\;\Conid{Ty}\;\rightarrow\;\Conid{Set}} which is interpreted as the list of \ensuremath{\Conid{HCont-NF}} for the spine,
i.e. the iterated domains of a given type (E.g. the spine of \ensuremath{\Conid{A}\;\Rightarrow\;\Conid{B}\;\Rightarrow\;\Conid{Set}} is the context \ensuremath{\bullet\;\rhd\;\Conid{A}\;\rhd\;\Conid{B}}.
\begin{hscode}\SaveRestoreHook
\column{B}{@{}>{\hspre}l<{\hspost}@{}}%
\column{3}{@{}>{\hspre}l<{\hspost}@{}}%
\column{5}{@{}>{\hspre}l<{\hspost}@{}}%
\column{7}{@{}>{\hspre}l<{\hspost}@{}}%
\column{E}{@{}>{\hspre}l<{\hspost}@{}}%
\>[B]{}\Keyword{record}\;\Conid{HCont-NE}\;\Gamma\;\Keyword{where}{}\<[E]%
\\
\>[B]{}\hsindent{3}{}\<[3]%
\>[3]{}\Keyword{inductive}{}\<[E]%
\\
\>[B]{}\hsindent{3}{}\<[3]%
\>[3]{}\Keyword{field}{}\<[E]%
\\
\>[3]{}\hsindent{2}{}\<[5]%
\>[5]{}\Conid{S}\;\mathbin{:}\;\Conid{Var}\;\Gamma\;\Conid{A}\;\rightarrow\;\Conid{Set}{}\<[E]%
\\
\>[3]{}\hsindent{2}{}\<[5]%
\>[5]{}\Conid{P}\;\mathbin{:}\;\{\mskip1.5mu \Varid{x}\;\mathbin{:}\;\Conid{Var}\;\Gamma\;\Conid{A}\mskip1.5mu\}\;(\Varid{s}\;\mathbin{:}\;\Conid{S}\;\Varid{x})\;\rightarrow\;\Conid{Set}{}\<[E]%
\\
\>[3]{}\hsindent{2}{}\<[5]%
\>[5]{}\Conid{R}\;\mathbin{:}\;\{\mskip1.5mu \Varid{x}\;\mathbin{:}\;\Conid{Var}\;\Gamma\;\Conid{A}\mskip1.5mu\}\;\{\mskip1.5mu \Varid{s}\;\mathbin{:}\;\Conid{S}\;\Varid{x}\mskip1.5mu\}\;(\Varid{p}\;\mathbin{:}\;\Conid{P}\;\Varid{s})\;\rightarrow\;\Conid{HCont-SP}\;\Gamma\;\Conid{A}{}\<[E]%
\\[\blanklineskip]%
\>[B]{}\Keyword{data}\;\Conid{HCont-SP}\;\Keyword{where}{}\<[E]%
\\
\>[B]{}\hsindent{3}{}\<[3]%
\>[3]{}\epsilon\;{}\<[7]%
\>[7]{}\mathbin{:}\;\Conid{HCont-SP}\;\Gamma\;\Varid{set}{}\<[E]%
\\
\>[B]{}\hsindent{3}{}\<[3]%
\>[3]{}\Varid{\char95 ,\char95 }\;\mathbin{:}\;\Conid{HCont-NF}\;\Gamma\;\Conid{A}\;\rightarrow\;\Conid{HCont-SP}\;\Gamma\;\Conid{B}\;\rightarrow\;\Conid{HCont-SP}\;\Gamma\;(\Conid{A}\;\Rightarrow\;\Conid{B}){}\<[E]%
\ColumnHook
\end{hscode}\resethooks
In the example we define an element of
\ensuremath{\Conid{HCont}\;(\Varid{set}\;\Rightarrow\;\Varid{set})\;\Rightarrow\;(\Varid{set}\;\Rightarrow\;\Varid{set})} using \ensuremath{\Varid{lam}} twice reducing it to
\ensuremath{\Conid{HCont-NE}\;\Gamma_0} with \ensuremath{\Gamma_0\;\mathrel{=}\;\bullet\;\rhd\;\Varid{set}\;\Rightarrow\;\Varid{set}\;\rhd\;\Varid{set}}. Since there are no sums
involved \ensuremath{\Conid{S}} is always \ensuremath{\top}. We have two elements of \ensuremath{\Conid{Var}\;\Gamma\;\Conid{A}} in both
cases \ensuremath{\Conid{P}\;\Varid{x}\;\mathrel{=}\;\top} since there is exactly one top-level occurence. In the
first case with \ensuremath{\Conid{A}\;\mathrel{=}\;\Varid{set}} there is no recursion since the domain
context s empty, while in the second one we need to model
\ensuremath{\Conid{B'}\;\Conid{F}\;\Conid{X}\;\mathrel{=}\;\Conid{F}\;(\Conid{F}\;\Conid{X})} which proceeds recursively.

We can define:
\begin{hscode}\SaveRestoreHook
\column{B}{@{}>{\hspre}l<{\hspost}@{}}%
\column{7}{@{}>{\hspre}l<{\hspost}@{}}%
\column{E}{@{}>{\hspre}l<{\hspost}@{}}%
\>[B]{}\Varid{[\char95 ]ne}\;\mathbin{:}\;\Conid{HCont-NE}\;\Gamma\;\rightarrow\;[\mskip1.5mu \;\Gamma\;\Varid{]C}\;\rightarrow\;\Conid{Set}{}\<[E]%
\\
\>[B]{}\Varid{[\char95 ]sp}\;\mathbin{:}\;\Conid{HCont-SP}\;\Gamma\;\Conid{A}\;\rightarrow\;[\mskip1.5mu \;\Gamma\;\Varid{]C}\;\rightarrow\;[\mskip1.5mu \;\Conid{A}\;\Varid{]T}{}\<[E]%
\\
\>[B]{}\Varid{[\char95 ]nf}\;\mathbin{:}\;\Conid{HCont-NF}\;\Gamma\;\Conid{A}\;\rightarrow\;[\mskip1.5mu \;\Gamma\;\Varid{]C}\;\rightarrow\;[\mskip1.5mu \;\Conid{A}\;\Varid{]T}{}\<[E]%
\\
\>[B]{}\Varid{[\char95 ]H}\;{}\<[7]%
\>[7]{}\mathbin{:}\;\Conid{HCont}\;\Conid{A}\;\rightarrow\;[\mskip1.5mu \;\Conid{A}\;\Varid{]T}{}\<[E]%
\ColumnHook
\end{hscode}\resethooks
where \ensuremath{\Varid{[\char95 ]T}\;\mathbin{:}\;\Conid{Ty}\;\rightarrow\;\Conid{Set1}} and \ensuremath{\Varid{[\char95 ]C}\;\mathbin{:}\;\Conid{Con}\;\rightarrow\;\Conid{Set1}} are the interpretation
of types in the intended model.

we have just started our investigation, and hope to be able to extend
the standard results of containers to these higher kinded
containers. In particular we would like to show
\begin{itemize}
\item that higher containers give rise to higher hereditary functors,
 
\item that higher containers for a model of simply typed
  $\lambda$-calculus,
  
\item provide a notion of higher container morphisms which are
  complete,

\item construct initial algebras and terminal coalgebras of
  endo-containers,

\item reinterpret and extend the results from \cite{altenkirch2001representations}.
 
\end{itemize}

\section*{Acknowledgements}
\label{sec:acknowledgements}

This development is based on discussions the first author had with
H{\aa}kan Gylterud.


\bibliographystyle{plain}
%\bibliographystyle{alpha}
%\bibliographystyle{unsrt}
%\bibliographystyle{abbrv}
\bibliography{references}

%------------------------------------------------------------------------------

\end{document}
